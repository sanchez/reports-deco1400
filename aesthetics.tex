% !TeX spellcheck = en_US
% !TeX root = main.tex

\section{Aesthetics}
\subsection{Style Guide}
% Summarise the general aesthetic you've chosen and your design intentions
% How does your visual aesthetic engage your specific target audience?
% Visualise (with contextual examples):
% - Which color scheme did you use? Include HEX codes
% - Describe your text treatments. Include font names, sizes and weights
% - Describe any image or icon treatments
% - Describe any button/link treatments (e.g. hovering on a link)
% Rationalise the design choices you've made, relating to the design principles from the lectures

\subsubsection{Color Scheme}
\cite{colors}
% https://www.verywellmind.com/color-psychology-2795824
\subsubsection{Font Families}
\subsubsection{Font Weights and Sizes}
\subsubsection{Links}
% hovering over a navbar link has the border-width animated. Meant to show the link "activating" on hover
\subsubsection{Bouncing}
\subsubsection{Code Blocks}
% spaced and colored so easily caught to the eye when scrolling
\subsubsection{Background}
% images and colors are washed out or dark so that the user is drawn to the content first and the background is just an after thought
\subsubsection{Smooth Hovering}
% all hover effects are meant to appear smooth and with no sudden movements. Site is for people just beginning and they dont want to be scared off. Animations not too slow that it is unusable for the experienced person

\subsection{Aesthetics User Testing}
% Reflect on the success of the Aesthetics User Testing design activity you did in Week 9
% Include screenshots of the mockups you used for the activity
% Include the testing plan you developed for the activity
% What feedback did you get and how did it inform your aesthetic decisions?