% !TeX spellcheck = en_US
% !TeX root = main.tex

\section{Getting to Know Stakeholders}
\subsection{Target Audience}
% Who is the specific target audience in relation to the brief?
% Include a few personas that represent different, archetypal users
% What are some common traits you've identified from your personas?
% What implications does this audience have for the design?
It is always essential to pick a target audience for a project before beginning, this will help to identify relevant content and ideas that should or should not be used. For this project the target audience is centred around people just beginning to learn about code or beginning to enter a professional environment where their code or actions will be marked and versioned. These sorts of people at typically around 18 to 22 years old and have some basic experience with computers and know how to work their way around the system. However they are still incredibly knew to the environment and if content is not broken down in an understandable way then they can be easily confused and lose interest.

\subsubsection{Goals}
With the target audience specified it can now be import to outline goals that should be followed to ensure the target audience is engaged.
\todo[inline]{Get all the references}
\begin{description}
	\item[Consumable Chunks:] The target audience require information to be broken down into consumable chunks in which they can learn, reflect, and then use before moving on to the next chunk.
	\item[Direct Content Flow:] Content chunks should also logically flow from one section to the next without large gaps in presentation. Not following this goal will ultimately result in the consumer losing track and becoming disinterested.
	\item[Simplistic Graphics:] Following design patterns released by large web driven companies, such as Google and Facebook, content should be shown in a way that is not overcrowded and can be confusing. If graphics are used to illustrate a point they should be vector based to help provide a more defining and clean look.
	\item[Defined Website Theme:] In order to engage the users and not make them lose interest, the website must have a clear and engaging theme that is applied site wide. This will help to reduce confusion about different sections of the site.
	\item[Clean and Polished:] The site as a whole should be clean and usable site. This means performance should remain consistent and have an expected behaviour across the entire site. As well all graphics should be presented in a high quality and consistent manor, this can be achieved using vector graphics and rendering on the client side. Finally the site should function across all modern browsers and devices, as well as having accessibility support.
\end{description}


\subsection{Chosen Educational Content}
% What is your chosen educational content?
% Why is this educational content an interesting choice for the target audience?
For this website, the goal is to be a resource for people wanting to learn about git the version source control system. Git allows people to track and mark changes they are making to code as well as easily allowing other people to integrate and edit the code together. Git fits well into the target audience because it is becoming more and more essential programming jobs and is becoming assumed knowledge for every programmer. Therefore people just starting in industry might not have knowledge about this tool and therefore need to upskill quickly and efficiently.

\subsection{Chosen Story}
% What is your chosen story and genre?
% Why is this story an interesting choice for the website's visuals and theme in relation to the target audience?
\todo[inline]{Find a story}