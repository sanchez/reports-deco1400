% !TeX spellcheck = en_US
% !TeX root = main.tex

\section{Getting to Know Stakeholders}
\subsection{Target Audience}
% Who is the specific target audience in relation to the brief?
% Include a few personas that represent different, archetypal users
% What are some common traits you've identified from your personas?
% What implications does this audience have for the design?
It is always essential to pick a target audience for a project before beginning, this will help to identify relevant content and ideas that should or should not be used. For this project the target audience is centred around people just beginning to learn about code or beginning to enter a professional environment where their code or actions will be marked and versioned. These sorts of people at typically around 18 to 22 years old and have some basic experience with computers and know how to work their way around the system. However they are still incredibly new to the environment and if content is not broken down in an understandable way then they can be easily confused and lose interest.

\subsubsection{Personas}
\persona{Jane}{21}{Female}{\begin{itemize}
	\item Has been applying for jobs but her skill set isn't wide enough to make it past the first round	
\end{itemize}}{\begin{itemize}
	\item She gives up incredibly easily	
\end{itemize}}

\persona{David}{23}{Male}{\begin{itemize}
	\item Wants to move into a more software development role in his company	
\end{itemize}}{\begin{itemize}
	\item Easily distracted and ends up losing track of current task	
\end{itemize}}

\persona{John}{17}{Male}{\begin{itemize}
	\item Has an assignment coming up and has heard how Git can help him keep track and control his code	
\end{itemize}}{\begin{itemize}
	\item Has mentality of ``Why learn this if my old thing can do a similar job''	
\end{itemize}}

\persona{Bob}{19}{Male}{\begin{itemize}
	\item The rest of his team wants to use Git for an assignment and he is the only one in the team that doesn't know it	
\end{itemize}}{\begin{itemize}
	\item He doesn't want to learn, he is just being pressured by the rest of his team	
\end{itemize}}

\subsubsection{Goals}\label{sec:goals}
With the target audience specified it can now be import to outline goals that should be followed to ensure the target audience is engaged.
% todo[inline]{Get all the references}
\begin{description}
	\item[Consumable Chunks:] The target audience require information to be broken down into consumable chunks in which they can learn, reflect, and then use before moving on to the next chunk.
	\item[Direct Content Flow:] Content chunks should also logically flow from one section to the next without large gaps in presentation. Not following this goal will ultimately result in the consumer losing track and becoming disinterested.
	\item[Simplistic Graphics:] Following design patterns released by large web driven companies, such as Google and Facebook, content should be shown in a way that is not overcrowded and can be confusing. If graphics are used to illustrate a point they should be vector based to help provide a more defining and clean look.
	\item[Defined Website Theme:] In order to engage the users and not make them lose interest, the website must have a clear and engaging theme that is applied site wide. This will help to reduce confusion about different sections of the site.
	\item[Clean and Polished:] The site as a whole should be clean and usable site. This means performance should remain consistent and have an expected behaviour across the entire site. As well all graphics should be presented in a high quality and consistent manor, this can be achieved using vector graphics and rendering on the client side. Finally the site should function across all modern browsers and devices, as well as having accessibility support.
\end{description}


\subsection{Chosen Educational Content}\label{sec:content}
% What is your chosen educational content?
% Why is this educational content an interesting choice for the target audience?
For this website, the goal is to be a resource for people wanting to learn about \gls{git} the version source control system. \Gls{git} allows people to track and mark changes they are making to code as well as easily allowing other people to integrate and edit the code together. \Gls{git} fits well into the target audience because it is becoming more and more essential programming jobs and is becoming assumed knowledge for every programmer. Therefore people just starting in industry might not have knowledge about this tool and therefore need to upskill quickly and efficiently.

\subsection{Chosen Story}
% What is your chosen story and genre?
% Why is this story an interesting choice for the website's visuals and theme in relation to the target audience?
\subsubsection{Reference Character}
The chosen character which this story is based off is Dr. Henry Walton Jones Jr, or more commonly referred to as Indiana Jones. This character fits into the target audience as most people growing up either watched one of the many movies or side television shows, or just know about the character from their friends. Indiana Jones also has a number of books and novels covering his adventures which helps in finding source materials.

\subsubsection{The Story}\label{sec:TheStory}
The story being portrayed is based around Indiana Jones and his adventures through caves and tombs. This adventure has Indiana Jones in search of the sacred ``Scroll of Lost Truths'', he will venture out to wild jungles in search of a cave where this scroll is said to be hidden amongst a long set of winding and mind boggling tunnels. The ``Scroll of Lost Truths'' is said to provide its wielder with the ability to recover any lost information and the history behind the information. In order to help Indiana Jones through the tunnels, he will be keeping a record of all the actions he completes as well as the effects these actions have on the mystical tunnel.

\subsubsection{How It Relates}
% tunnels -> git branches
% history behind the artifact -> git history (i.e. history on the walls)
% scroll of lost truths -> single source of truth -> master branch / git
In order to tie the story (Section~\ref{sec:TheStory}) and the educational material (Section~\ref{sec:content}), many references need to be drawn in the story. Achieving this will help to create an understanding and flowing story line that will entice and draw users in to follow through to the end.\\\\

The different ties between \gls{git} and Indiana Jones are:
\begin{description}
	\item[The History:] In order to tie together the whole history aspect of \gls{git}, remarks will be made between the ancient carvings in the walls of the tunnel and how \gls{git} keeps track of all the code changes created.
	\item[Branching:] The tunnels will be used to illustrate how in \gls{git} you can branch out and take a different route if it requires.
	\item[Single Source of Truth:] \Gls{git} can sometimes be referred to as the single source of truth. The scroll will demonstrate that by showing that once you master \gls{git}, you have the ability to traverse through history and create new history.
\end{description}





